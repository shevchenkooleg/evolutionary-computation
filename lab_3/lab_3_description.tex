\documentclass[12pt]{article}
\usepackage[T2A]{fontenc}
\usepackage[utf8]{inputenc}
\usepackage[english, russian]{babel}
\usepackage{amsmath, amssymb, amsthm}
\usepackage{geometry}
\usepackage{graphicx}
\usepackage{float}
\usepackage{booktabs}
\usepackage{array}
\usepackage{multirow}
\usepackage{caption}
\usepackage{algorithm}
\usepackage{algpseudocode}
\usepackage{subcaption}
\usepackage{pifont}
\usepackage{microtype}
\newcommand{\cmark}{\ding{51}} % ✓
\newcommand{\xmark}{\ding{55}} % ✗

\geometry{a4paper, left=20mm, right=15mm, top=20mm, bottom=20mm}

\title{Лабораторная работа №3 \\ Исследование алгоритма оптимизации роем частиц для минимизации функции Растригина}
\author{Студент: Шевченко О.В. \\ Группа: 09.04.01-ПОВа-з25}
\date{\today}

\begin{document}

\maketitle

\section{Цель работы}
\label{sec:goal}

Данная лабораторная работа посвящена практическому изучению \textbf{алгоритма оптимизации роем частиц} (Particle Swarm Optimization, PSO) как метода глобальной оптимизации для многоэкстремальных функций. В качестве тестового объекта выбрана \textbf{функция Растригина} -- классический пример сложного ландшафта с большим количеством локальных минимумов.

Цели данной работы включают:

\begin{enumerate}
    \item \textbf{Практическая реализация:} Освоение процедуры программирования основных компонентов PSO: представления частицы, инициализации роя, обновления скоростей и позиций, обработки границ.
    \item \textbf{Исследование параметров:} Систематическое изучение влияния ключевых параметров PSO (размера роя, коэффициента инерции, когнитивного и социального коэффициентов, максимальной скорости) на скорость сходимости, точность и стабильность поиска глобального минимума.
    \item \textbf{Визуальный анализ:} Для случая $n=2$ -- построение и анализ динамики роя в пространстве поиска на фоне линий уровня целевой функции, визуализация процесса оптимизации.
    \item \textbf{Исследование масштабируемости:} Анализ эффективности PSO при увеличении размерности задачи.
    \item \textbf{Сравнительный анализ:} Осмысление сильных и слабых сторон роевого интеллекта в контексте задачи глобальной оптимизации многоэкстремальной функции.
\end{enumerate}

Результатом работы должно стать глубокое понимание принципов роевого интеллекта, умение настраивать параметры PSO для конкретной задачи и критически оценивать область его применимости.

\section{Теоретическая часть}
\label{sec:theory}

\subsection{Постановка задачи оптимизации}
\label{subsec:problem}

Требуется найти глобальный минимум \textbf{функции Растригина} для $n$ переменных:
\begin{equation}
\label{eq:rastrigin}
f(\mathbf{x}) = A n + \sum_{i=1}^{n} \left[ x_i^2 - A \cos(2 \pi x_i) \right],
\end{equation}
где:
\begin{itemize}
    \item $\mathbf{x} = (x_1, x_2, \dots, x_n) \in \mathbb{R}^n$ -- вектор оптимизируемых переменных,
    \item $A$ -- положительный параметр (стандартное значение $A = 10$),
    \item Область поиска: $x_i \in [-5.12, 5.12]$, $i = 1, \dots, n$ (общепринятый интервал для тестирования).
\end{itemize}

\textbf{Свойства функции:}
\begin{itemize}
    \item \textbf{Глобальный минимум:} $f(\mathbf{x}^*) = 0$, достигается в единственной точке $\mathbf{x}^* = (0, 0, \dots, 0)$.
    \item \textbf{Многомодальность:} Функция имеет $\prod_{i=1}^n m_i$ локальных минимумов, где $m_i$ -- число локальных минимумов по координате $x_i$ (при $A=10$, $m_i \approx 10$ на интервале $[-5.12, 5.12]$). Для $n=2$ это более 50 локальных минимумов.
    \item \textbf{Сложность:} Регулярная осциллирующая структура создаёт «ловушки» для локальных методов поиска, что делает её отличным тестом для глобальных оптимизаторов, таких как PSO.
\end{itemize}

\subsection{Основы алгоритма оптимизации роем частиц}
\label{subsec:pso_basics}

Алгоритм оптимизации роем частиц (PSO) -- это эвристический метод глобальной оптимизации, вдохновленный коллективным поведением птичьих стай и рыбьих косяков. В PSO потенциальные решения, называемые \textbf{частицами}, перемещаются в пространстве поиска согласно простым математическим формулам, учитывающим их собственную лучшую позицию и лучшую позицию среди всех частиц.

\subsubsection{Основные понятия}
\begin{itemize}
    \item \textbf{Частица:} Одно возможное решение задачи, представляющее собой точку в пространстве поиска. Каждая частица $i$ имеет:
    \begin{itemize}
        \item \textbf{Позицию:} $\mathbf{x}_i = (x_{i1}, x_{i2}, \dots, x_{in})$
        \item \textbf{Скорость:} $\mathbf{v}_i = (v_{i1}, v_{i2}, \dots, v_{in})$
        \item \textbf{Лучшая личная позиция:} $\mathbf{p}_i = (p_{i1}, p_{i2}, \dots, p_{in})$ -- наилучшая позиция, которую частица обнаружила за все время.
        \item \textbf{Значение функции:} $f(\mathbf{x}_i)$
    \end{itemize}
    \item \textbf{Рой:} Совокупность $N$ частиц: $S = \{\mathbf{x}_1, \mathbf{x}_2, \dots, \mathbf{x}_N\}$.
    \item \textbf{Глобальная лучшая позиция:} $\mathbf{g} = (g_1, g_2, \dots, g_n)$ -- наилучшая позиция, обнаруженная всеми частицами роя.
\end{itemize}

\subsubsection{Математическая модель PSO}
\label{subsec:pso_model}

Обновление скорости и позиции частицы на каждой итерации происходит по следующим формулам:

\begin{align}
\mathbf{v}_i^{(t+1)} &= w \cdot \mathbf{v}_i^{(t)} + c_1 r_1 (\mathbf{p}_i - \mathbf{x}_i^{(t)}) + c_2 r_2 (\mathbf{g} - \mathbf{x}_i^{(t)}) \label{eq:velocity_update} \\
\mathbf{x}_i^{(t+1)} &= \mathbf{x}_i^{(t)} + \mathbf{v}_i^{(t+1)} \label{eq:position_update}
\end{align}

где:
\begin{itemize}
    \item $w$ -- коэффициент инерции (inertia weight), контролирующий влияние предыдущей скорости.
    \item $c_1$ -- когнитивный коэффициент (cognitive coefficient), определяющий склонность частицы возвращаться к своей лучшей найденной позиции.
    \item $c_2$ -- социальный коэффициент (social coefficient), определяющий склонность частицы следовать к глобальной лучшей позиции.
    \item $r_1, r_2 \sim U(0,1)$ -- случайные числа из равномерного распределения.
    \item $t$ -- номер итерации.
\end{itemize}

\subsubsection{Ключевые параметры PSO и их влияние}
\label{subsec:pso_params}

\begin{itemize}
    \item \textbf{Размер роя $N$:} Определяет количество параллельно исследующих частиц. Слишком маленький рой ($<10$) может не охватить сложный ландшафт, слишком большой ($>100$) увеличивает вычислительные затраты.
    \item \textbf{Коэффициент инерции $w$:} 
    \begin{itemize}
        \item $w > 1$: Частицы ускоряются, могут вылететь за границы области.
        \item $0 < w < 1$: Частицы замедляются, сходимость к оптимуму.
        \item $w < 0$: Частицы движутся в обратном направлении.
    \end{itemize}
    \item \textbf{Когнитивный коэффициент $c_1$:} Определяет индивидуальный интеллект частицы. Большие значения способствуют исследованию (exploration).
    \item \textbf{Социальный коэффициент $c_2$:} Определяет коллективный интеллект роя. Большие значения способствуют использованию найденных решений (exploitation).
    \item \textbf{Максимальная скорость $v_{\max}$:} Ограничивает скорость частиц для предотвращения неконтролируемого движения.
\end{itemize}

\subsubsection{Стратегии обработки границ}
\label{subsec:boundary_strategies}

При выходе частицы за границы области поиска применяются различные стратегии:
\begin{itemize}
    \item \textbf{Отражение (reflect):} Частица отражается от границы, меняя знак соответствующей компоненты скорости.
    \item \textbf{Поглощение (absorb):} Частица останавливается на границе, скорость в направлении границы обнуляется.
    \item \textbf{Циклическое возвращение (cyclic):} Частица появляется с противоположной стороны области (тор).
\end{itemize}

\subsection{Связь PSO с задачей минимизации Растригина}
\label{subsec:pso_rastrigin}

Для успешной минимизации функции Растригина с PSO необходимо:
\begin{enumerate}
    \item Обеспечить достаточное начальное разнообразие роя для покрытия области с множеством локальных минимумов.
    \item Подобрать баланс между исследованием (нахождение новых областей) и использованием (уточнение найденных решений).
    \item Учитывать регулярную осциллирующую структуру функции при выборе параметров скорости.
    \item Реализовать механизмы предотвращения преждевременной сходимости к локальным минимумам.
\end{enumerate}

\section{Практическая часть}
\label{sec:practical}

\subsection{Этап 1: Подготовка данных и среды}
\label{subsec:setup}

\subsubsection{Параметры задачи и алгоритма}
\begin{enumerate}
    \item \textbf{Функция Растригина:}
    \begin{itemize}
        \item Параметр $A = 10$.
        \item Область определения: $x_i \in [-5.12, 5.12]$.
        \item Глобальный минимум: $f(\mathbf{0}) = 0$.
    \end{itemize}

    \item \textbf{Базовые параметры PSO:}
    \begin{align*}
        \text{Размер роя:} & \quad N = 30 \\
        \text{Максимальное число итераций:} & \quad G_{\max} = 100 \\
        \text{Коэффициент инерции:} & \quad w = 0.7 \\
        \text{Когнитивный коэффициент:} & \quad c_1 = 1.5 \\
        \text{Социальный коэффициент:} & \quad c_2 = 1.5 \\
        \text{Максимальная скорость:} & \quad v_{\max} = 1.0 \\
        \text{Стратегия обработки границ:} & \quad \text{reflect}
    \end{align*}

    \item \textbf{Критерии остановки:}
    \begin{align*}
        \text{Достигнута точность:} & \quad f_{\text{best}} < \varepsilon_f = 10^{-4} \\
        \text{Превышено число итераций:} & \quad t > G_{\max} \\
        \text{Стагнация:} & \quad \text{Лучшее значение не улучшалось 20 итераций}.
    \end{align*}
\end{enumerate}

\subsubsection{Инструменты реализации}
Для реализации алгоритма и визуализации результатов используется язык Python 3.x с библиотеками:
\begin{itemize}
    \item \texttt{numpy} - для векторных вычислений и генерации случайных чисел.
    \item \texttt{matplotlib} - для построения графиков и визуализации в 2D/3D.
    \item \texttt{seaborn} - для улучшения визуализации данных.
\end{itemize}

\subsection{Этап 2: Реализация алгоритма PSO}
\label{subsec:implementation}

\subsubsection{Псевдокод алгоритма}
\begin{algorithm}[H]
\caption{Алгоритм оптимизации роем частиц для минимизации функции Растригина}
\begin{algorithmic}[1]
\State \textbf{Вход:} $n, N, G_{\max}, w, c_1, c_2, v_{\max}, \varepsilon_f$
\State \textbf{Выход:} $\mathbf{g}, f_{\text{best}}, \text{history}$
\State
\Function{PSO}{}
    \State $\text{swarm} \gets \text{ИнициализироватьРой}(N, n, -5.12, 5.12)$
    \State $\mathbf{g} \gets \text{None}, f_{\text{best}} \gets \infty$
    \State $\text{history} \gets \{\}$
    \State $t \gets 0, \text{stagnation\_counter} \gets 0$
    
    \While{$t < G_{\max}$ \textbf{and} $f_{\text{best}} > \varepsilon_f$}
        \ForAll{$\text{particle} \in \text{swarm}$}
            \State $f_{\text{curr}} \gets f(\text{particle.position})$
            \If{$f_{\text{curr}} < \text{particle.best\_value}$}
                \State $\text{particle.best\_value} \gets f_{\text{curr}}$
                \State $\text{particle.best\_position} \gets \text{particle.position}$
            \EndIf
            \If{$f_{\text{curr}} < f_{\text{best}}$}
                \State $f_{\text{best}} \gets f_{\text{curr}}$
                \State $\mathbf{g} \gets \text{particle.position}$
            \EndIf
        \EndFor
        
        \State $\text{СохранитьСтатистику}(history, t, f_{\text{best}}, \text{swarm})$
        
        \ForAll{$\text{particle} \in \text{swarm}$}
            \State $\mathbf{r}_1, \mathbf{r}_2 \sim U(0,1)^n$
            \State $\text{particle.velocity} \gets w \cdot \text{particle.velocity}$
            \State \quad $+ c_1 \cdot \mathbf{r}_1 \odot (\text{particle.best\_position} - \text{particle.position})$
            \State \quad $+ c_2 \cdot \mathbf{r}_2 \odot (\mathbf{g} - \text{particle.position})$
            \State $\text{ОграничитьСкорость}(\text{particle.velocity}, v_{\max})$
            \State $\text{particle.position} \gets \text{particle.position} + \text{particle.velocity}$
            \State $\text{ОбработатьГраницы}(\text{particle}, -5.12, 5.12, \text{reflect})$
        \EndFor
        
        \If{$\text{УлучшенияНет}(history)$}
            \State $\text{stagnation\_counter} \gets \text{stagnation\_counter} + 1$
        \Else
            \State $\text{stagnation\_counter} \gets 0$
        \EndIf
        
        \If{$\text{stagnation\_counter} > 20$}
            \State \textbf{break}
        \EndIf
        
        \State $t \gets t + 1$
    \EndWhile
    
    \State \Return $\mathbf{g}, f_{\text{best}}, \text{history}$
\EndFunction
\end{algorithmic}
\end{algorithm}

\subsection{Этап 3: Эксперименты и анализ}
\label{subsec:experiments}

\subsubsection{Задание 3.1: Визуализация функции Растригина}
\label{subsubsec:rastrigin_visualization}

\begin{figure}[H]
\centering
\includegraphics[width=0.8\textwidth]{rastrigin_function_pso.png}
\caption{Функция Растригина для $n=2$: (а) 3D поверхность, (б) линии уровня}
\label{fig:rastrigin_visualization}
\end{figure}

На рисунке \ref{fig:rastrigin_visualization} представлена визуализация функции Растригина для двумерного случая. Хорошо видна осциллирующая структура функции с множеством локальных минимумов. Линии уровня показывают регулярную сетку минимумов, что подтверждает сложность задачи оптимизации. Глобальный минимум находится в точке $(0, 0)$.

\subsubsection{Задание 3.2: Базовый запуск PSO}
\label{subsubsec:base_run}

\begin{table}[H]
\centering
\begin{tabular}{|c|c|c|c|}
\hline
\textbf{Параметр} & \textbf{Значение} & \textbf{Параметр} & \textbf{Значение} \\ \hline
Размерность ($n$) & 2 & Коэффициент инерции ($w$) & 0.7 \\ \hline
Размер роя ($N$) & 30 & Когнитивный коэффициент ($c_1$) & 1.5 \\ \hline
Итераций & 100 & Социальный коэффициент ($c_2$) & 1.5 \\ \hline
\end{tabular}
\caption{Параметры базового запуска PSO}
\label{tab:base_params}
\end{table}

\textbf{Результаты базового запуска:}
\begin{itemize}
    \item Лучшее значение функции: $f_{\text{best}} = 2.4 \times 10^{-5}$
    \item Количество итераций до сходимости: 29
    \item Время выполнения: 0.02 секунды
    \item Сошелся: Да
\end{itemize}

Базовый запуск показал высокую эффективность PSO для минимизации функции Растригина в двумерном случае. Алгоритм быстро сошелся к глобальному минимуму с высокой точностью.

\subsubsection{Задание 3.3: Исследование влияния размера роя $N$}
\label{subsubsec:swarm_size}

\begin{enumerate}
    \item \textbf{Цель:} Определить оптимальный размер роя для эффективного поиска глобального минимума.
    \item \textbf{Параметры эксперимента:} $n=2$, $G_{\max}=100$, $w=0.7$, $c_1=1.5$, $c_2=1.5$, $v_{\max}=1.0$. Значения $N$: [10, 20, 30, 50, 100].
    \item \textbf{Результаты:}

    \begin{table}[H]
    \centering
    \begin{tabular}{|c|c|>{\centering\arraybackslash}p{2.5cm}|>{\centering\arraybackslash}p{2.5cm}|>{\centering\arraybackslash}p{7cm}|}
    \hline
    \textbf{N} & \textbf{$f_{\text{best}}$ (сред.)} & \textbf{Итерации (сред.)} & \textbf{Успешность, \%} & \textbf{Характер поиска} \\ \hline
    10 & 0.6635 & 65.0 & 33.3 & Недостаточное покрытие \\ \hline
    20 & 0.6633 & 62.0 & 33.3 & Локальные минимумы \\ \hline
    30 & $4.3 \times 10^{-5}$ & 38.0 & 100.0 & Оптимальный баланс \\ \hline
    50 & $6.2 \times 10^{-5}$ & 48.3 & 100.0 & Надежный поиск \\ \hline
    100 & $5.9 \times 10^{-5}$ & 32.7 & 100.0 & Быстрая сходимость \\ \hline
    \end{tabular}
    \caption{Влияние размера роя $N$ на эффективность PSO ($n=2$)}
    \label{tab:swarm_size_results}
    \end{table}

    \item \textbf{Анализ:}
    \begin{enumerate}
        \item Как видно из таблицы \ref{tab:swarm_size_results}, при $N=10$ и $N=20$ рой слишком мал для адекватного покрытия сложного ландшафта. Алгоритм часто «застревает» в локальных минимумах (успешность 33.3\%).
        \item При $N=30$ достигается хороший баланс: высокая успешность (100\%), хорошая скорость сходимости (38 итераций).
        \item При $N=50$ и $N=100$ алгоритм становится очень надежным (100\% успешность), но стоимость каждой итерации выше.
        \item \textbf{Вывод:} Для функции Растригина при $n=2$ оптимальный размер роя находится в диапазоне 30-50 частиц. Слишком маленький рой не обеспечивает достаточного разнообразия, слишком большой -- вычислительно неэффективен.
    \end{enumerate}
\end{enumerate}

\subsubsection{Задание 3.4: Исследование влияния коэффициента инерции $w$}
\label{subsubsec:inertia}

\begin{enumerate}
    \item \textbf{Цель:} Оценить роль коэффициента инерции в балансе между исследованием и использованием.
    \item \textbf{Параметры:} $n=2$, $N=30$, $c_1=1.5$, $c_2=1.5$, $v_{\max}=1.0$. Значения $w$: [0.4, 0.7, 0.9, 1.2].
    \item \textbf{Результаты:}

    \begin{table}[H]
    \centering
    \begin{tabular}{|c|c|>{\centering\arraybackslash}p{2.5cm}|>{\centering\arraybackslash}p{2.5cm}|>{\centering\arraybackslash}p{7cm}|}
    \hline
    \textbf{w} & \textbf{$f_{\text{best}}$ (сред.)} & \textbf{Итерации (сред.)} & \textbf{Успешность, \%} & \textbf{Характер поиска} \\ \hline
    0.4 & 0.3317 & 34.0 & 66.7 & Быстрая, но преждевременная сходимость \\ \hline
    0.7 & $3.7 \times 10^{-5}$ & 38.0 & 100.0 & Оптимальный баланс \\ \hline
    0.9 & 0.3326 & 65.7 & 33.3 & Медленная, застревает в локальных минимумах \\ \hline
    1.2 & 0.3350 & 59.7 & 0.0 & Нестабильный, хаотичный поиск \\ \hline
    \end{tabular}
    \caption{Влияние коэффициента инерции $w$ на эффективность PSO ($n=2$)}
    \label{tab:inertia_results}
    \end{table}

    \item \textbf{Анализ:}
    \begin{enumerate}
        \item Как показано в таблице \ref{tab:inertia_results}, слишком низкое значение $w=0.4$ приводит к быстрой, но преждевременной сходимости (успешность 66.7\%).
        \item Оптимальное значение $w=0.7$ обеспечивает баланс между исследованием новых областей и использованием найденных решений (100\% успешность).
        \item Высокие значения $w=0.9$ и $w=1.2$ приводят к нестабильному поведению: частицы слишком инерционны, плохо реагируют на лучшие позиции, что снижает эффективность поиска.
        \item \textbf{Вывод:} Коэффициент инерции критически важен для баланса exploration/exploitation. Рекомендуемое значение: $w=0.7-0.8$ для большинства задач.
    \end{enumerate}
\end{enumerate}

\subsubsection{Задание 3.5: Исследование влияния когнитивного коэффициента $c_1$}
\label{subsubsec:cognitive}

\begin{enumerate}
    \item \textbf{Цель:} Изучить влияние индивидуального опыта частиц на поисковые возможности алгоритма.
    \item \textbf{Параметры:} $n=2$, $N=30$, $w=0.7$, $c_2=1.5$, $v_{\max}=1.0$. Значения $c_1$: [0.5, 1.0, 1.5, 2.0, 2.5].
    \item \textbf{Результаты:}

    \begin{table}[H]
    \centering
    \begin{tabular}{|c|c|>{\centering\arraybackslash}p{2.5cm}|>{\centering\arraybackslash}p{2.5cm}|>{\centering\arraybackslash}p{7cm}|}
    \hline
    \textbf{$c_1$} & \textbf{$f_{\text{best}}$ (сред.)} & \textbf{Итерации (сред.)} & \textbf{Успешность, \%} & \textbf{Характер поиска} \\ \hline
    0.5 & 0.6633 & 55.0 & 33.3 & Недостаточный индивидуальный поиск \\ \hline
    1.0 & 0.6633 & 62.3 & 33.3 & Слабое притяжение к личным лучшим \\ \hline
    1.5 & 0.3317 & 44.3 & 66.7 & Умеренный индивидуальный поиск \\ \hline
    2.0 & 0.3317 & 54.7 & 66.7 & Активный индивидуальный поиск \\ \hline
    2.5 & 0.6633 & 74.7 & 33.3 & Излишний индивидуальный поиск \\ \hline
    \end{tabular}
    \caption{Влияние когнитивного коэффициента $c_1$ на эффективность PSO ($n=2$)}
    \label{tab:cognitive_results}
    \end{table}

    \item \textbf{Анализ:}
    \begin{enumerate}
        \item Как видно из таблицы \ref{tab:cognitive_results}, низкие значения $c_1$ (0.5, 1.0) приводят к недостаточному индивидуальному поиску. Частицы слабо возвращаются к своим лучшим позициям.
        \item Значения $c_1=1.5$ и $c_1=2.0$ показывают наилучшие результаты (успешность 66.7\%). Частицы активно исследуют окрестности своих лучших позиций.
        \item Слишком высокое значение $c_1=2.5$ приводит к излишнему индивидуальному поиску в ущерб коллективному интеллекту.
        \item \textbf{Вывод:} Когнитивный коэффициент должен быть достаточно большим для активного индивидуального поиска, но не чрезмерным. Рекомендуемое значение: $c_1=1.5-2.0$.
    \end{enumerate}
\end{enumerate}

\subsubsection{Задание 3.6: Исследование влияния социального коэффициента $c_2$}
\label{subsubsec:social}

\begin{enumerate}
    \item \textbf{Цель:} Изучить влияние коллективного опыта роя на поисковые возможности алгоритма.
    \item \textbf{Параметры:} $n=2$, $N=30$, $w=0.7$, $c_1=1.5$, $v_{\max}=1.0$. Значения $c_2$: [0.5, 1.0, 1.5, 2.0, 2.5].
    \item \textbf{Результаты:}

    \begin{table}[H]
    \centering
    \begin{tabular}{|c|c|>{\centering\arraybackslash}p{2.5cm}|>{\centering\arraybackslash}p{2.5cm}|>{\centering\arraybackslash}p{7cm}|}
    \hline
    \textbf{$c_2$} & \textbf{$f_{\text{best}}$ (сред.)} & \textbf{Итерации (сред.)} & \textbf{Успешность, \%} & \textbf{Характер поиска} \\ \hline
    0.5 & $6.4 \times 10^{-5}$ & 48.0 & 100.0 & Умеренное коллективное притяжение \\ \hline
    1.0 & $2.9 \times 10^{-5}$ & 39.7 & 100.0 & Сбалансированное коллективное притяжение \\ \hline
    1.5 & $3.5 \times 10^{-5}$ & 54.7 & 100.0 & Активное коллективное притяжение \\ \hline
    2.0 & $5.3 \times 10^{-5}$ & 31.7 & 100.0 & Сильное коллективное притяжение \\ \hline
    2.5 & 0.3317 & 51.3 & 66.7 & Излишнее коллективное притяжение \\ \hline
    \end{tabular}
    \caption{Влияние социального коэффициента $c_2$ на эффективность PSO ($n=2$)}
    \label{tab:social_results}
    \end{table}

    \item \textbf{Анализ:}
    \begin{enumerate}
        \item Как показано в таблице \ref{tab:social_results}, значения $c_2$ от 0.5 до 2.0 показывают высокую успешность (100\%).
        \item Наилучшая скорость сходимости достигается при $c_2=2.0$ (31.7 итераций в среднем).
        \item Слишком высокое значение $c_2=2.5$ приводит к преждевременной сходимости к субоптимальным решениям (успешность 66.7\%).
        \item \textbf{Вывод:} Социальный коэффициент критически важен для быстрой сходимости PSO. Рекомендуемое значение: $c_2=1.5-2.0$ для эффективного использования коллективного интеллекта.
    \end{enumerate}
\end{enumerate}

\subsubsection{Задание 3.7: Исследование влияния максимальной скорости $v_{\max}$}
\label{subsubsec:vmax}

\begin{enumerate}
    \item \textbf{Цель:} Определить оптимальное ограничение скорости для эффективного поиска.
    \item \textbf{Параметры:} $n=2$, $N=30$, $w=0.7$, $c_1=1.5$, $c_2=1.5$. Значения $v_{\max}$: [0.1, 0.5, 1.0, 2.0, 5.0].
    \item \textbf{Результаты:}

    \begin{table}[H]
    \centering
    \begin{tabular}{|c|c|c|c|c|}
    \hline
    \textbf{$v_{\max}$} & \textbf{$f_{\text{best}}$ (сред.)} & \textbf{Итерации (сред.)} & \textbf{Успешность, \%} & \textbf{Характер поиска} \\ \hline
    0.1 & 0.0030 & 80.7 & 66.7 & Медленный, точный поиск \\ \hline
    0.5 & 0.3317 & 47.3 & 66.7 & Умеренная скорость \\ \hline
    1.0 & 0.6633 & 67.3 & 33.3 & Стандартная скорость \\ \hline
    2.0 & 0.6633 & 68.7 & 33.3 & Высокая скорость \\ \hline
    5.0 & $4.3 \times 10^{-5}$ & 44.0 & 100.0 & Очень высокая скорость \\ \hline
    \end{tabular}
    \caption{Влияние максимальной скорости $v_{\max}$ на эффективность PSO ($n=2$)}
    \label{tab:vmax_results}
    \end{table}

    \item \textbf{Анализ:}
    \begin{enumerate}
        \item Как видно из таблицы \ref{tab:vmax_results}, слишком низкие скорости ($v_{\max}=0.1$) приводят к медленному, но точному поиску.
        \item Оптимальное значение $v_{\max}=5.0$ показывает наилучшие результаты (100\% успешность, 44 итерации).
        \item Промежуточные значения (0.5-2.0) демонстрируют снижение эффективности.
        \item \textbf{Вывод:} Для функции Растригина с диапазоном $[-5.12, 5.12]$ оптимальная максимальная скорость составляет около 5.0 (примерно 50\% от диапазона). Это позволяет частицам эффективно преодолевать локальные минимумы.
    \end{enumerate}
\end{enumerate}

\subsubsection{Задание 3.8: Сравнение баланса $c_1$ и $c_2$}
\label{subsubsec:c1_c2_balance}

\begin{enumerate}
    \item \textbf{Цель:} Исследовать влияние баланса между индивидуальным и коллективным интеллектом.
    \item \textbf{Конфигурации:}
    \begin{itemize}
        \item Личный опыт: $c_1=2.5$, $c_2=0.5$
        \item Коллективный опыт: $c_1=0.5$, $c_2=2.5$
        \item Баланс: $c_1=1.5$, $c_2=1.5$
        \item Активный поиск: $c_1=2.0$, $c_2=2.0$
    \end{itemize}
    \item \textbf{Результаты:}

    \begin{table}[H]
    \centering
    \begin{tabular}{|c|c|>{\centering\arraybackslash}p{2.5cm}|>{\centering\arraybackslash}p{2.5cm}|>{\centering\arraybackslash}p{4cm}|}
    \hline
    \textbf{Конфигурация} & \textbf{$f_{\text{best}}$ (сред.)} & \textbf{Итерации (сред.)} & \textbf{Успешность, \%} & \textbf{Характер поиска} \\ \hline
    Личный опыт & 0.0019 & 45.7 & 66.7 & Индивидуальный поиск \\ \hline
    Коллективный опыт & 0.6633 & 62.0 & 33.3 & Коллективный поиск \\ \hline
    Баланс & 0.9950 & 53.3 & 33.3 & Сбалансированный \\ \hline
    Активный поиск & 0.6633 & 74.3 & 33.3 & Агрессивный поиск \\ \hline
    \end{tabular}
    \caption{Сравнение баланса $c_1$ и $c_2$ ($n=2$, $N=30$, $w=0.7$)}
    \label{tab:balance_results}
    \end{table}

    \item \textbf{Анализ:}
    \begin{enumerate}
        \item Конфигурация «Личный опыт» показала наилучший результат (успешность 66.7\%), что указывает на важность индивидуального поиска для функции Растригина.
        \item Конфигурация «Коллективный опыт» оказалась неэффективной, что связано с преждевременной сходимостью к локальным минимумам.
        \item Сбалансированные конфигурации также показали низкую эффективность.
        \item \textbf{Вывод:} Для многомодальных функций типа Растригина более важным оказывается индивидуальный поиск (высокий $c_1$), позволяющий частицам активно исследовать окрестности своих лучших позиций и преодолевать локальные минимумы.
    \end{enumerate}
\end{enumerate}

\subsubsection{Задание 3.9: Визуализация работы PSO в 2D ($n=2$)}
\label{subsubsec:visualization}

\begin{figure}[H]
\centering
\includegraphics[width=0.9\textwidth]{pso_particle_trajectories.png}
\caption{Траектории частиц PSO в пространстве поиска (линии уровня функции Растригина)}
\label{fig:pso_trajectories}
\end{figure}

\begin{figure}[H]
\centering
\includegraphics[width=0.9\textwidth]{pso_convergence_curves.png}
\caption{Динамика сходимости PSO: (а) лучшее значение функции, (б) среднее значение функции, (в) разнообразие роя, (г) скорости частиц}
\label{fig:pso_convergence}
\end{figure}

\textbf{Ключевые наблюдения из визуализации:}
\begin{enumerate}
    \item На рисунке \ref{fig:pso_trajectories} видно, как частицы начинают со случайных позиций и постепенно сходятся к глобальному минимуму.
    \item На рисунке \ref{fig:pso_convergence} наблюдаются характерные фазы работы PSO:
    \begin{itemize}
        \item \textbf{Фаза 1 (итерации 1-10):} Быстрое улучшение $f_{\text{best}}$, высокое разнообразие роя, высокие скорости частиц.
        \item \textbf{Фаза 2 (итерации 10-20):} Замедление улучшений, снижение разнообразия и скоростей.
        \item \textbf{Фаза 3 (итерации 20-30):} Стагнация, низкое разнообразие, частицы сходятся к оптимальной области.
    \end{itemize}
    \item PSO демонстрирует эффективное сочетание глобального поиска (в начале) и локального уточнения (в конце).
\end{enumerate}

\subsubsection{Задание 3.10: Исследование влияния размерности задачи}
\label{subsubsec:dimensions}

\begin{enumerate}
    \item \textbf{Цель:} Изучить масштабируемость PSO при увеличении размерности пространства поиска.
    \item \textbf{Параметры:} $N=50$, $G_{\max}=200$, $w=0.7$, $c_1=1.5$, $c_2=1.5$, $v_{\max}=1.0$. Размерности: $n = [2, 5, 10, 20]$.
    \item \textbf{Результаты:}

    \begin{table}[H]
    \centering
    \begin{tabular}{|c|c|c|c|c|}
    \hline
    \textbf{n} & \textbf{$f_{\text{best}}$ (сред.)} & \textbf{Итерации (сред.)} & \textbf{Успешность, \%} & \textbf{Относительная ошибка} \\ \hline
    2 & 0.3317 & 51.3 & 66.7 & 0.0332 \\ \hline
    5 & 3.9798 & 93.0 & 0.0 & 0.3980 \\ \hline
    10 & 14.2646 & 100.0 & 0.0 & 1.4265 \\ \hline
    20 & 41.5164 & 100.0 & 0.0 & 4.1516 \\ \hline
    \end{tabular}
\caption{Эффективность PSO в зависимости от размерности $n$}
\label{tab:dimension_results}
\end{table}

    \item \textbf{Анализ:}
    \begin{enumerate}
        \item Как показано в таблице \ref{tab:dimension_results}, с ростом размерности эффективность PSO резко падает.
        \item При $n=2$ алгоритм успешно находит решение с высокой точностью (относительная ошибка 3.32\%).
        \item При $n=5$ PSO уже не достигает глобального минимума (успешность 0\%), среднее значение функции 3.98.
        \item При $n=10$ и $n=20$ алгоритм находит решения, лишь незначительно лучше случайных.
        \item \textbf{Вывод:} PSO страдает от «проклятия размерности». Для задач высокой размерности ($n > 5$) необходимы:
        \begin{itemize}
            \item Существенное увеличение размера роя.
            \item Более сложные стратегии обновления параметров.
            \item Возможна адаптация коэффициентов в процессе поиска.
            \item Гибридные подходы (PSO + локальные методы).
        \end{itemize}
    \end{enumerate}
\end{enumerate}

\subsubsection{Сводный анализ результатов и рекомендации}
\label{subsubsec:summary_analysis}

\begin{table}[H]
\centering
\renewcommand{\arraystretch}{1.3}
\begin{tabular}{|p{4.5cm}|p{4cm}|p{6cm}|}
\hline
\textbf{Исследуемый параметр} & \textbf{Рекомендуемый диапазон} & \textbf{Влияние на поиск} \\ \hline
Размер роя $N$ & $30 - 50$ для $n=2-5$ & Определяет начальное покрытие пространства. Больше $N$ = надежнее, но требует больше вычислений. \\ \hline
Коэффициент инерции $w$ & $0.7 - 0.8$ & Контролирует баланс между исследованием и использованием. Уменьшение со временем улучшает сходимость. \\ \hline
Когнитивный коэффициент $c_1$ & $1.5 - 2.5$ & Определяет индивидуальный интеллект. Важен для преодоления локальных минимумов. \\ \hline
Социальный коэффициент $c_2$ & $1.5 - 2.0$ & Определяет коллективный интеллект. Обеспечивает быструю сходимость. \\ \hline
Максимальная скорость $v_{\max}$ & $20-50\%$ диапазона & Предотвращает неконтролируемое движение. Для Растригина: $v_{\max} \approx 5.0$. \\ \hline
Баланс $c_1$/$c_2$ & $c_1 > c_2$ для многомодальных функций & Для функций с многими локальными минимумами важнее индивидуальный поиск. \\ \hline
\end{tabular}
\caption{Сводные рекомендации по настройке параметров PSO для функции Растригина}
\label{tab:final_recommendations}
\end{table}

\section{Общие выводы}
\label{sec:conclusion}

В ходе лабораторной работы был успешно реализован и исследован алгоритм оптимизации роем частиц для минимизации многоэкстремальной функции Растригина. Получены следующие основные результаты и выводы:

\begin{enumerate}
    \item \textbf{Эффективность PSO для многомодальных задач:} Алгоритм PSO подтвердил свою способность эффективно находить глобальный минимум функции Растригина в низкоразмерных случаях ($n \leq 3$). Механизм коллективного интеллекта позволяет частицам «перепрыгивать» через локальные минимумы.

    \item \textbf{Критическая важность параметров:} Эффективность PSO сильно зависит от правильного выбора параметров:
    \begin{itemize}
        \item \textbf{Размер роя $N$} должен быть достаточным для покрытия сложного ландшафта (30-50 частиц для $n=2$).
        \item \textbf{Коэффициент инерции $w$} критически важен для баланса exploration/exploitation (оптимум 0.7-0.8).
        \item \textbf{Когнитивный коэффициент $c_1$} более важен для многомодальных функций, чем социальный коэффициент $c_2$.
        \item \textbf{Максимальная скорость $v_{\max}$} должна быть достаточно большой для преодоления локальных минимумов.
    \end{itemize}

    \item \textbf{Проблема высокой размерности:} Как и другие методы глобальной оптимизации, PSO страдает от «проклятия размерности». Эффективность резко падает при $n > 5$, что требует специальных модификаций для высокоразмерных задач.

    \item \textbf{Сравнительный анализ методов оптимизации (ЛР №1-№3):}
    
    В ходе выполнения трех лабораторных работ были исследованы три принципиально различных подхода к оптимизации. Каждый метод имеет свои сильные и слабые стороны, которые определяют область его применения.

\begin{table}[H]
    \centering
    \renewcommand{\arraystretch}{1.2}
    \begin{tabular}{|p{5.5cm}|p{5.5cm}|p{5.5cm}|}
    \hline
    \textbf{Градиентный спуск (ЛР №1)} & \textbf{Генетический алгоритм (ЛР №2)} & \textbf{Алгоритм PSO (ЛР №3)} \\ \hline
    \textbf{+} \textbf{Быстрая локальная сходимость}: Экспоненциальная сходимость для выпуклых задач & \textbf{+} \textbf{Глобальный поиск}: Способен преодолевать локальные минимумы & \textbf{+} \textbf{Баланс поиска}: Эффективное сочетание exploration и exploitation \\ \hline
    \textbf{+} \textbf{Теоретическая обоснованность}: Четкие условия сходимости, теоретические оценки & \textbf{+} \textbf{Робастность}: Работает с разрывными, негладкими функциями & \textbf{+} \textbf{Простота реализации}: Минимальное количество операторов \\ \hline
    \textbf{+} \textbf{Эффективность вычислений}: Низкая вычислительная сложность на итерацию & \textbf{+} \textbf{Параллелизм}: Независимая оценка особей позволяет распараллеливание & \textbf{+} \textbf{Естественная память}: Частицы запоминают личные и глобальные лучшие позиции \\ \hline
    \textbf{--} \textbf{Локальность}: Застревает в локальных минимумах многомодальных функций & \textbf{--} \textbf{Медленная сходимость}: Требует многих итераций для точного решения & \textbf{--} \textbf{Преждевременная сходимость}: Склонен сходиться к субоптимумам \\ \hline
    \textbf{--} \textbf{Требование гладкости}: Необходимо вычисление градиента & \textbf{--} \textbf{Сложность настройки}: Множество параметров (вероятности, турниры и т.д.) & \textbf{--} \textbf{Проблемы размерности}: Эффективность падает при $n > 10$ \\ \hline
    \textbf{--} \textbf{Чувствительность к параметрам}: Критическая зависимость от шага $\alpha$ & \textbf{--} \textbf{Вычислительная стоимость}: Высокая стоимость оценки популяции & \textbf{--} \textbf{Ограниченная теория}: Меньше теоретических гарантий сходимости \\ \hline
    \end{tabular}
    \caption{Сравнительные характеристики методов оптимизации}
    \label{tab:comparison_all_methods}
\end{table}





\item \textbf{Область применения PSO:} Алгоритм оптимизации роем частиц наиболее эффективен для:
    \begin{itemize}
        \item Непрерывных задач оптимизации с гладкими или умеренно нерегулярными функциями.
        \item Задач с умеренной размерностью ($n \leq 10$).
        \item Задач, где важна быстрая сходимость к приемлемому решению.
        \item Задач, допускающих распараллеливание (независимая оценка частиц).
    \end{itemize}

    \item \textbf{Направления дальнейшего исследования:} Для повышения эффективности базового PSO можно рассмотреть:
    \begin{itemize}
        \item Адаптацию параметров ($w$, $c_1$, $c_2$) в процессе поиска.
        \item Гибридные алгоритмы (PSO + локальный поиск).
        \item Многороевые модификации.
        \item Стратегии поддержания разнообразия (нишевые PSO).
        \item Параллельные реализации для ускорения вычислений.
    \end{itemize}

    \item \textbf{Практические рекомендации для функции Растригина:}
    \begin{enumerate}
        \item Для $n=2$: $N=30$, $w=0.7$, $c_1=2.0$, $c_2=1.5$, $v_{\max}=5.0$.
        \item Для $n=5$: Увеличить размер роя до 50-100 частиц.
        \item Использовать стратегию отражения границ для поддержания активности поиска.
        \item Внедрить механизм уменьшения $w$ со временем: $w(t) = w_{\text{начальное}} \times (0.99)^t$.
        \item Для мониторинга сходимости отслеживать разнообразие роя и среднюю скорость частиц.
    \end{enumerate}
\end{enumerate}

Таким образом, лабораторная работа позволила не только освоить принципы алгоритма оптимизации роем частиц, но и получить практический опыт настройки его параметров для сложной многомодальной функции. PSO продемонстрировал себя как эффективный и интуитивно понятный метод глобальной оптимизации, особенно для задач умеренной размерности. Однако, как и другие метаэвристические методы, он требует тщательного подбора параметров и понимания его ограничений, связанных с масштабируемостью на высокоразмерные задачи.

\end{document}